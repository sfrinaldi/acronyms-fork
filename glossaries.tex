%! Author = srinaldi
%! Date = 2024-09-23

% For recording terms and definitions (separate from acronyms) to be used in latex generating repositories or for reference.
% Use the following command for adding a new term to the glossary:
% \newglossaryentry{term}{type={term},name={term},description={term description}}

\newglossaryentry{assembly}
{
    type={term},
    name={assembly},
    description={A functional subdivision of a component, consisting of parts or subassemblies, which perform functions necessary for the operation of the component as a whole. Examples: regulator assembly, power amplifier assembly, gyro assembly, etc.}
}
\newglossaryentry{cheater plug}
{
    type={term},
    name={cheater plug},
    description={An Alternating Current \gls{AC} plug adapter used to connect a three-pronged plug to a two-pronged AC socket. It can be used to separate the ground wire from the socket for testing purposes.}
}
\newglossaryentry{component}
{
    type={term},
    name={component},
    description={A functional subdivision of a system, generally a self-contained combination of assemblies performing a function necessary for the system's operation. Examples: power supply, transmitter, gyro package, etc.}
}
\newglossaryentry{conductive material}{type={term},name={conductive material},description={A material that has a surface resistivity of <105 ohms per square or a volume resistivity <104 ohms-cm.}}
\newglossaryentry{electrostatic discharge (ESD)}{type={term},name={electrostatic discharge},description={A transfer of electrostatic charge between bodies at different electrostatic potentials caused by direct contact or induced by an electrostatic field.}}
\newglossaryentry{ESD-Protective Material}{type={term},name={ESD-Protective Material},description={Material capable of one or more of the following functions: limiting the generation of static electricity; safely dissipating electrostatic charges over its surface or volume; or providing shielding from ESD spark discharge or electrostatic fields.}}
\newglossaryentry{ESD Protected Area}{type={term},name={ESD Protected Area},description={An area that is constructed and equipped with the necessary ESD-protective materials and equipment to limit ESD voltage below the sensitivity level of ESDS items handled therein. This may include benches, rooms or buildings.}}
\newglossaryentry{Kit}{type={term},name={Kit},description={A prepared package of parts with instructions for assembly and/or wiring a component or chassis.}}
\newglossaryentry{Operator}{type={term},name={Operator},description={An individual who is trained and certified to perform tasks in an ESD protected area.}}
\newglossaryentry{Part}{type={term},name={Part},description={An element of a component, assembly, or subassembly which is not normally subject to further subdivision or disassembly without destruction of its designed use, e.g., a module, Integrated Circuit (IC), resistor, etc.}}
\newglossaryentry{Soft Ground}{type={term},name={Soft Ground},description={A connection to ground through impedance sufficiently high to limit current flow to safe levels for personnel (normally 5 milliamperes). Impedance needed for a soft ground is dependent upon the voltage levels which could be contacted by personnel near the ground. By this definition a hard ground protected by a functional GFCI is considered a soft ground.}}
\newglossaryentry{Static Dissipative}{type={term},name={Static Dissipative},description={A property of a material having surface resistivity ≥105 but <1012 ohms per square or a volume resistivity ≥104 but <1011 ohms-cm.}}
\newglossaryentry{Surface Resistivity}{type={term},name={Surface Resistivity},description={The surface resistivity is an inverse measure of the conductivity of a material. Surface resistivity of a material is numerically equal to the surface resistance between two electrodes forming opposite sides of a square. The size of the square is immaterial. Surface resistivity applies to both surfaces and materials with constant volume conductivity and has the value of ohms per square.}}
\newglossaryentry{Triboelectric}{type={term},name={Triboelectric},description={Pertaining to electricity generated by friction.}}
\newglossaryentry{Electrostatic Field}{type={term},name={Electrostatic Field},description={A voltage gradient between an electrostatically charged surface and another surface of a different electrostatic potential.}}
\newglossaryentry{Ground}{type={term},name={Ground},description={A mass such as earth, a ship, or a vehicle hull, capable of supplying or accepting a large electrical charge.}}
\newglossaryentry{Groundable Point}{type={term},name={Groundable Point},description={Any point with low impedance to ground where grounding may be attached. Usually it is the common point ground.}}
\newglossaryentry{Hard Ground}{type={term},name={Hard Ground},description={A connection to earth ground either directly or through low impedance.}}
\newglossaryentry{Insulative Material}{type={term},name={Insulative Material},description={A material having a surface resistivity ≥1012 ohms/square or a volume resistivity ≥1011 ohms-cm.}}

